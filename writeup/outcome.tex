\section{Outcome}
\label{sec:correctness-and-security}

As we saw, we could use (verifiable) secret sharing, collaborative evaluation of constant-depth circuits (in particular collaborative XOR), pseudorandom generators as a basis for a new protocol the securely and secretly evaluates a given function. These protocols are known to work in presence of dishonest players, as long as these players are a minority. Furthermore, they work in a constant number of rounds using a polynomial amount of communication. These techniques are then used to collaboratively compute a garbled circuit (with garbled inputs), that is in turn locally evaluated by each player individually.

Note that we didn't exactly specify which technique should be used for these actions. We just ``ensured'' that \emph{there are} techniques that we can use.

That, in turn, simplifies proofs, since \emph{we can assume} that for these building blocks we have come constant-round polynomial-communication protocols. The proof then builds upon these assumptions.

Even if we won't go through the proof, let us briefly restate what we have seen by now: It is possible to collaboratively evaluate a function $f$ on inputs $x_1,\dots,x_n$ such that player $i$ supplies $x_i$ and that all the inputs stay private to the respective players. This is even more surprising if one takes into account that we did \emph{not} bound the size of the circuit in some way.

Proving the above statement requires very sophisticated arguments, which is why we won't do an analysis of the proof. For these reasons, we omitted a more formal in-depth-treatment of the protocol. Inclined readers find a more detailed proof in \cite{Rogaway:1991:RCS:888502}.

But reader beware: the learning curve for understanding these proofs is not small. A variety of definitions, formalizations, lemmata and theorems has to be studied before one can even start to think about the proof. For a relatively up-to-date treatment of garbled circuits that introduces necessary concepts, \cite{bellare-hoang-rogaway-garbling-schemes} might be a good start.

%%% Local Variables: 
%%% mode: latex
%%% TeX-master: "seminar"
%%% End: 